\documentclass[a4paper,10pt]{article}
\usepackage[utf8]{inputenc}


\usepackage{geometry}
\geometry{margin=1in}
\usepackage{setspace}
%\singlespacing
\onehalfspacing
%\doublespacing
%\setstretch{1.1}

% Title Page
\title{POD-based deflation techniques for two-phase flow simulation in large and highly heterogeneous porous media.}
\author{Gabriela Diaz}


\begin{document}

\maketitle
\section*{Introduction}

\hspace{0.5cm}Reservoir simulation involves the solution of a set of partial differential equations
which, often, lead to a system of linear equations. When simulating two phases, the solution of 
the pressure equation is the most time-consuming part, especially in the cases where the system is 
large or ill-conditioned. Furthermore, if it is required to compute a large number of simulations,
the solution of this problem becomes expensive. Therefore, techniques to improve the 
simulation speed are required.\\
Reduced Order Models (ROM) have been studied to reduce the dimension of the system, by capturing 
relevant information and using it to project high-dimensional data into a lower-dimension 
space \cite{Vermeulen04,Kerschen05,Pasetto16,Schilders08,Quarteroni14,Carlberg15}. These methods 
show that essential information of the system can be captured by computing a set of basis 
functions based on solutions of the system. These solutions are known as 'snapshots', where the relevant information of the system is stored for later use.\\
Proper Orthogonal Decomposition (POD) is a ROM method that has been studied in recent years to 
accelerate the solution of the pressure equation, resulting from reservoir simulation 
\cite{Astrid11,Mark06, Mark09,Cardoso09,Heijn04,Doren06}, among other applications. 
With POD procedures, only few basis functions are necessary. The basis is such that it contains most of the variation of the original system \cite{Cardoso09,Kerschen05}. Hence, the system can be 
represented in terms of this basis. 
Once the basis is obtained, the POD method can be used in different ways. For the 
solution of a large-scale system, Markovinovic et al. \cite{Mark06} propose the use of POD techniques to compute a good 
initial guess that accelerates the solution of an iterative method. The
solution of the problem in the small-scale domain and the projection back to the large-scale system
is also approached by Astrid et al. \cite{Astrid11}. Pasetto et al. \cite{Pasetto16} propose the use of this basis for the construction of a preconditioner. Diaz Cortes et al. \cite{Diaz17} propose the use of the POD basis as deflation matrix.\\
If the system is not only large but also ill-conditioned (containing large variations in the coefficients), some Krylov subspace iterative
methods are also used \footnote{Given a linear system $\mathbf{A}\mathbf{x}=\mathbf{b}$, and the initial residual $\mathbf{r}^0=\mathbf{b}-\mathbf{A}\mathbf{x}^0$, with $\mathbf{x}^0$ an initial guess of $\mathbf{x}$, we define the Krylov subspace as
$\mathcal{K}_k(\mathbf{A},\mathbf{r}^0)=span\{\mathbf{r}^0,\mathbf{A}\mathbf{r}^0,\dots,\mathbf{A}^{k-1}\mathbf{r}^0\}$ \cite{Saad00}. That is, the set of linear combinations of powers of $\mathbf{A}$ times $\mathbf{r}^0$. }.  The speed of convergence of these 
methods depends on the condition number and the right-hand side of the system. If the condition 
number is very large, some techniques can be implemented to reduce it. Commonly, preconditioners (matrices modifying the original system) are used. These matrices, in 
general, are cheap to compute and they cluster the eigenvalues of the original system, which transforsm the system into a better conditioned one.\\ In recent years, deflation techniques have been approached for the acceleration of the convergence of Krylov subspace methods \cite{Vuik99,Vuik02,Tang08,Tang09,Kahl17}. For a good performance of this 
method, a deflation subspace needs to be found, such that, the smallest eigenvalues of the 
system are no longer present and the iterative method is be accelerated. In this work, we use POD techniques to 
construct the above-mentioned deflation subspace. We use this methodology to solve two-phase flow problems in large-scale, highly heterogeneous porous media. \\
\small
 \bibliographystyle{unsrt}
 \newpage
 \bibliography{/home/wagm/cortes/Localdisk/Research/bib/research}
\end{document}          
