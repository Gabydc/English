\documentclass[12pt]{article}
\usepackage{pifont}
\usepackage{a4wide}
\usepackage[utf8x]{inputenc}
\usepackage{amsmath}
\usepackage{amsfonts}
\usepackage{mathrsfs}
%\usepackage{natbib}
\usepackage{graphicx} % figuras
%\usepackage[export]{adjustbox} % loads also graphicx
\usepackage{float}
\usepackage[font=footnotesize]{caption}
\usepackage{wrapfig}
\usepackage{authblk}
\usepackage{subfigure}

\usepackage{amssymb}
\usepackage{latexsym}
\usepackage[sort&compress]{natbib}

\setlength{\parindent}{1em} 

\topmargin=-2pt
\title{POD-based deflation techniques for the solution of two-phase flow problems in large and highly heterogeneous porous media.}

\author[1]{G. B. Diaz Cortes}  
%\author[1]{C. Vuik} 
%\author[2]{J. D. Jansen} 
\date{}
%\affil[1]{Department of Applied Mathematics, TU Delft}
%\affil[2]{Department of Geoscience \& Engineering, TU Delft}
\renewcommand\Authands{ and }
%\date{April 2017}


\begin{document}
% \thispagestyle{empty}
% \noindent
% \begin{center}
% {\Large \sc DELFT UNIVERSITY OF TECHNOLOGY}
% \\
% \vspace{3cm}
% {\large \sc REPORT 17-01}\\[4ex]
% {\large \sc On POD-based Deflation Vectors for DPCG applied to porous media problems.}\\[4ex]
% {\large \sc G. B. Diaz Cortes, C. Vuik, J. D. Jansen}\\
% \vfill
% {\tt ISSN 1389-6520}\\[2ex]
% {\tt Reports of the Delft Institute of Applied Mathematics}\\[2ex]
% {\tt Delft 2017}
% \end{center}
% \pagebreak
% \thispagestyle{empty}
% \vspace*{\fill}
% \noindent
% \hspace*{-0,3cm}Copyright~~~\Pisymbol{psy}{227}~~~2017 by Delft Institute of Applied Mathematics, Delft, \mbox{The Netherlands.}
% \\[2ex]
% No part of the Journal may be reproduced, stored in a retrieval system, or
% transmitted, in any form or by any means, electronic, mechanical, photocopying,
% recording, or otherwise, without the prior written permission from Delft Institute of
% Applied Mathematics, Delft University of Technology, The
% Netherlands. 
% % newpage, title etc.
% \setcounter{page}{1}
\newpage
\maketitle
\begin{abstract}

\hspace{0.5cm}Simulation of two-phase flow through highly heterogeneous porous media results in ill-conditioned large systems of linear equations for the pressure when using, e.g., sequential procedures.
      Solving the resulting linear system can be particularly time-consuming. Therefore, there have been extensive efforts finding effective ways to address this issue.\par
      Iterative methods, together with preconditioning techniques \cite{Saad03,Benzi02}, are the most commonly chosen to solve this kind of problem. In the literature, we can also find Reduced Order Models (ROM) \cite{Astrid11,Mark06,Pasetto16} and deflation methods \cite{Saad00,Vuik99} where system information is reused to find a good approximation more quickly. For the deflation techniques, an optimal selection of deflation vectors is crucial for a good performance. These vectors are collected mainly via recycling deflation (vectors containing previous solutions); subdomain deflation
      (vectors constructed based on geometric properties); and restriction and prolongation matrices, obtained with multigrid and multilevel methods.\par
      The construction of deflated vectors based on information captured with ROM, in particular, Proper Orthogonal Decomposition (POD) was presented by Diaz et al. \cite{Diaz17,Diaz16_E}. 
      The goal of this work is to further explore the possibilities of combining POD and deflation techniques to speed-up the convergence of iterative methods. We propose selecting deflation vectors based on ROM methods; in particular, we focous on Proper Orthogonal Decomposition (POD). \par
     The convergence properties of the resulting POD-based deflation method are studied for an incompressible two-phase flow problem in a highly heterogeneous porous medium. We compare the number of iterations required to solve the above-mentioned problem using the Conjugate Gradient method preconditioned with Incomplete Cholesky (ICCG), against the deflated (DICCG) version of the same method.\par
     The efficiency of the method is illustrated with the SPE 10 benchmark problem, presenting a contrast in permeability coefficients of $\mathcal{O}(10^7)$ and containing $\mathcal{O}(10^6)$ cells. For this problem, the DICCG method requires only $\sim 30\%$ of the number of ICCG iterations. 



\end{abstract}
 \newpage
 \bibliographystyle{unsrt}
 \newpage
 \bibliography{/home/wagm/cortes/Localdisk/Research/bib/research}


\end{document}
