\documentclass[a4paper,10pt]{report}
\usepackage[utf8]{inputenc}




\begin{document}

\section*{What are the advantages and disadvantages of MOOCs?}
by Gabriela Berenice Diaz Cortes (EAP3, Wed).
\section*{Introduction}
The growth in Internet use and digital resources has led to the digital learning era. 
Informative videos, interactive software, multimedia apps and, in recent times, MOOCs, 
Massive Online Open Courses, have gained popularity.\\
In 2011, a free artificial-intelligence course was offered by Standford University that attracted
160,000 students from around the world. From that time onwards, many universities have become \footnote{could this
also be 'have been'?} interested in 
introducing these internet-based teaching programs\cite{waldrop13}.\\
Nowadays, in platforms like Cousera or EdX we can find courses from top world universities,
such as Harvard, the MIT or TU Delft. 
These courses comprise a wide range of subjects, from science, engineering and technology courses, that
are the most popular, to courses in the field of management, humanities, and arts that have been growing in recent years.
The courses count with a large number of digital resources such as documents, video lectures and discussion forums. 
Most of the courses are free and a single lecture can 
have a very large impact around the world. This way of teaching also allows the instructor to gain 
more experience, reputation, and teaching skills. \\
However, the interaction between the instructor and the students is minimal, there is rarely any education
credits offered for this kind of courses. Sometimes, it is very time-consuming to produce lectures and 
return costs are limited, furthermore, the completion rates are very low\cite{stark14}. \\
In this essay, I will give an overview of the pros and cons of this learning method.\\\\


Professors always want to innovate to improve\footnote{could this also be 'to innovate for the improvement of teaching'?} 
teaching and some of them
want to bring top universities-quality teaching to people who 
would never otherwise be able to study at such universities.
Daphne Koller, Profesor at Standford, wanted to promote ‘flipping’ a teaching technique 
in which students listen to lectures at home and
 do exercises in class. \\
In 2007 Profesor Andrew Ng in Stanford started
 a project to post online free lecture videos and handouts for ten of 
Stanford’s most popular engineering courses. 
After this, he noticed the impact of online courses.
With the ideas introduced by Ng, Koller made videos of 8–10-minute segments
 separated by pauses in which students had to answer questions or to solve problems. 
Later, she noticed that interaction between the students would be useful.
This started an online discussion forum in social-networking sites such as Facebook.
 Koller and Ng combined their achievements and started to work on a software platform for
 discussion forums and videos for an online course.
 With this platform they started an artificial-intelligence course 
for which they got 160,000 people registered from 195 countries, this marked 
the beginning of the MOOCs era.\\
MOOCs courses typically integrate social networking and are
characterized by the provision of online resources. MOOCs are
often facilitated by an acknowledged expert in the field. In 2013
only 13\% of all higher education institutions offered a MOOC, but
43\% plan to offer MOOCs by 2016. \\Some educators surveyed believe that MOOCs 
could complement the education offered by higher education institutions.

MOOCs have a significant number of advantages. 
One of the greatest benefits of MOOCs is their accessibility. 
These courses are usually low cost or free. They have flexibility of access, for example, 
for students who do not have access to higher education, because they lack financial 
resources or they have difficult working hours. The courses can be accessed anytime and anywhere.
The courses can be used by any number of students and are not limited to
college students, also professionals, younger students and the general public can 
participate as there is no need for prerequisites.
For younger students, they offer a way to know if they are interested in a
subject without large investment (tuition fees).\\
The students who take part in a MOOC develop lifelong learning skills because they are
encouraged to be aware of their own learning, and usually they develop an interest
in continuing their professional development. 
Student engagement can  be enhanced if the instructors recognize the learning styles of
students and establish good teaching strategies.
The challenges of MOOCs, as new technology, and a large 
numbers of students may encourage pedagogical development.

MOOCs have disadvantages as well. The accessibility 
of these courses, for example, may lead to some problems. Some people argue that  
MOOCs might not be well suited to all students, maybe only to highly motivated students with 
good studying skills, who could benefit from MOOCs. This theory is supported by the low rate 
of completion (around 4\%).
It is also thought that not all disciplines are
suitable for online studying.\\
The flexibility of access could also have negative consequences. For students, attending a MOOC 
can make it difficult to find a group of students
to find encouragement and endure long enough to pass the course. The online format
and the possibility of working at home, or anywhere, can be a distracting factor 
and could influence the effectiveness of the course.
For teachers, giving a MOOC course can be very challenging. 
The course material may need to be constantly
updated and this entail considerable time and effort. 
Detecting plagiarism or cheating could be very difficult too.\\
Furthermore, they think that there is a lack of personal
connection which is very important for teaching-learning processes.
Some students might take the course just out of curiosity and even if they learn from it,
they will not be engaged in it\footnote{with it?}.\\
Finally, the studies performed online are not always accepted as formal studies, 
which can affect the professional lives of students. 
\newpage
\section*{Conclusion}
Higher education has changed over the course of time and constantly adapts to culture and 
technology.\\
Nowadays, there is an increasing demand for higher education, but the number of institutions is not 
increasing at the same rate. MOOCs offer an alternative allowing a large number of
students to enroll in popular courses at top universities around the world. There are great expectations
about the possibilities provided by MOOCS and a large number of universities are looking forward to developing their own courses.\\
Even if a lot of people can benefit from these courses, there are still issues to solve. 
The low completion rates show that it is not easy to all the students to always complete the courses.
It is thought that the main problem is the poor interaction between the teacher and the students, 
and between students, and as they are open to the general public, the poor background of the participants 
could lead to failure.\\
For teachers, this could be very time-consuming and the benefits could be smaller compared with
the effort put in. Therefore, teachers need to consider if it really is worth it.\\
This is a new teaching method and it has to be further investigated to
obtain all the benefits from it. Research into this topic could lead to 
very useful teaching methods and might also help to develop better tools for 
 conventional teaching methods. 

\bibliographystyle{unsrt}
\bibliography{bibliography}
\end{document}          
