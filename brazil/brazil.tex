\documentclass[a4paper,10pt]{report}
\usepackage[utf8]{inputenc}

% Title Page
\title{}
\author{}


\begin{document}

Football in Brazil has had an important place in the lives of the poor working
class, the elites, and the government. (Eduardo Galeano. "Soccer: Opiate of the People?")
\\ Football had
social implications, as well as distinct class and race divisions, from the very beginning.
This sport shifted into what is now known as the poor people’s sport(Marcos Natali. "The Realm of the Possible: Remebering Brazilian Futebol.")
Football became an outlet of expression as well as a tool used to
form Brazilian identity. But it was also a symbol of exclusion and racism. “Futebol arte”
became a style specific to Brazil and it was believed, whether true or not, that Brazilians
possessed a uniqueness that differed from any other country (Marcos Natali. "The Realm of the Possible: Remebering Brazilian Futebol." )
\\In the later years it came to
symbolize different facets including modernity, it represented national identity and racial
acceptance for a large portion of Afro-Brazilians struggling to find peace in a racist
environment.
When football was just taking root in Brazil, the number of players and spectators was
rather small due to the exclusionary rules associated with it. Slavery was ingrained and it
wasn’t until 1888 when it was formally abolished that Brazil saw its movement toward
industrialization. Industrialization also gave rise to
factories and factory workers who were either former slaves or descendants of slaves.
\\Factory owners saw football as a means to create factory loyalty and disrupt any sense of
worker solidarity. (Rogerio Daflon, and Teo Ballvé. "The Beautiful Game? Race and Class in Brazilian Soccer." ).
Therefore teams were crafted to
benefit the owners by keeping the workers busy and happy. 
Now that more and more working
class men were allowed to play, the shift to the poor people’s sport became visible.
For instance, even though participation was increasing among the working class.
It was eventually acknowledged that having
black players on club and national teams were beneficial. One of the
first black players for the Brazilian team, Leônidas da Silva, became a huge football star
and ignited the idea that having people of color on the national team is an advantage, not
a problem. 
 Leônidas da Silva made a living out of playing football and earned not only respect
from Brazil but also around the world. For him, it was a way to climb up the social
ladder. For the working classes they were role models and an example to what can be
achieved(Rogerio Daflon and Teo Ballvé 25).
The local government was supported by volunteering and charity programs, with international funders
to provide support for local, football based development projects. (Griesbeck 14. The world cup Effect)\\
Tickets available for $15 USD$ to Brazilians. 100,000 tickets for free to builders working in stadiums and to 
for the socially disadvantaged. (FIFA webpage)\\
From the 2010 FIFA World Cup there was a positive impact that the sport left on South African youth.
Prior to the tournament, high levels of vandalism, bullying and exclusion were prevalent in Hillbrow, 
one of Johannesburg’s most infamous areas.  After the Dutch team sponsored the renovation of a 
football field in the heart of the neighborhood, interactions among the children seemed to change.  
On the wall, there are rules written that the children are obligated to follow; they include respect, 
fair play and social involvement” (Iob, Emilie. “South Africa Struggles to Maintain World Cup Legacy.”).\\
Typical festivities, cultural and
artistic events – that unveil a
country’s identity
greater national and international
visibility.(Sustainable Brazil, 
Social and Economic Impacts of the 2014 World Cup, Ernst & Young Terco)\\
\section*{Tourism}
According to government figures, 1 million foreign tourists visited Brazil during the month-long event, far exceeding its pre-Cup projection of 600,000 visitors coming to the country from abroad.
About 3 million Brazilians traveled around the country during the event, just short of the expected 3.1 million.
Additionally, according to the government, of the million foreign visitors, "95\% of them said they intend to return." (CNN, Brazil claims 'victory' in World Cup)
Apart from Rio de Janeiro and some cities in the Northeast, Brazil is not a major destination for tourists. 

The street vendor would like to extend the sales to foreigners 
who would be in Brazil during the World Cup or the Olympics, so he/she would put on some effort 
in trying to learn some English.

In order to promote the access to the learning of a foreign language, 
several initiatives have been taken. 
Online and classroom courses have been offered to high school students, 
taxi drivers, maids and other professionals who will be directly in 
touch with the foreign tourists coming to Brazil. Even employees working 
at the Municipal Market of São Paulo (Mercadão Municipal) have started 
to take English classes in order to meet the demand of foreign tourists.
São Paulo city Information available in both English and Portuguese. 
This is the case of shopping malls, subways and even buses. (http://thebrazilbusiness.com/)
\end{document}          
