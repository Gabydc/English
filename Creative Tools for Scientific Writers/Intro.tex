\documentclass[12pt]{article}
\usepackage[utf8x]{inputenc}
\usepackage{amsmath}
\usepackage{amsfonts}
\usepackage{mathrsfs}
\usepackage{natbib}
\usepackage{graphicx} % figuras
\usepackage[export]{adjustbox} % loads also graphicx
\usepackage{float}
\usepackage[font=footnotesize]{caption}
\usepackage{wrapfig}
\usepackage{authblk}
\usepackage{subfigure}
\usepackage{pifont}
\usepackage{a4wide}
\usepackage[ margin=2in]{geometry}
\topmargin=-2pt

\numberwithin{equation}{section}
%\usepackage{lipsum}
\topmargin=-2pt
\title{Creative Tools for Scientific Writing, \\Preparatory assignment 1}

\author{G. B. Diaz Cortes}  
% \affil[1]{Department of Applied Mathematics, TU Delft}
% \affil[2]{Department of Geoscience \& Engineering, TU Delft}
%\renewcommand\Authands{ and }
\date{November 2016}
\begin{document}



% Title Page

\maketitle


  \section{Introduction.}
  \hspace{0.5cm}Often, most computational time in the simulation of multiphase flow through porous media is taken 
  up by the
solution of the pressure equation. This involves solving large systems of linear equations as 
part of the iterative solution of the time and space discretized governing nonlinear partial differential 
equations. The time spent in solving the linear systems depends on the size of the problem and the 
variations of permeability within the medium. Solution of large-cale problems or problems with extreme contrasts in the
permeability values may lead to very large computing times. 
Iterative methods are known to be the best option to solve the above mentioned problems. However, sometimes the iterative methods are not 
sufficient to solve these problems in a reasonable amount of time and finding a way to accelerate the convergence of these methods becomes neccesary. \\
Preconditioners and Reduced Order Models (ROM) have been studied as tools to improve computational efficiency \cite{Barone09,Kala14} for problems involving large variations of permeability. Among ROM methods,
Proper Orthogonal Decomposition (POD) has been investigated in \cite{Mark06,Pasetto16,Carlberg15} with encouraging results. 
The use of a POD-based preconditioner for the acceleration of the solution is proposed by Astrid et al.
\cite{Astrid11} to solve the pressure equation resulting from two-phase reservoir simulation, and by Pasetto 
et al. \cite{Pasetto16} for groundwater flow models. 
The POD method requires the computation of a series of 'snapshots' which are solutions of the problem with slightly 
different parameters or well inputs. 
Once the snapshots are computed, the POD method is used to obtain a set of basis 
vectors that capture the most relevant features of the system. These basis vectors can be used later to speed-up the subsequent simulations.\\
Problems with a high contrast between the permeability coefficients are sometimes approached through the 
use of deflation techniques; see, e.g., \cite{Vuik99}. The use of deflation techniques involves the search 
of good deflation vectors, which are usually problem-dependent. 
Following the ideas of \cite{Astrid11,Mark06,Pasetto16,Carlberg15}, we propose the use of POD of many snapshots 
to capture the system's behavior. Furthermore, we combine this technique with deflation to accelerate the convergence of iterative methods. \\
In this work, the basis obtained with POD is 
studied as an alternative choice of deflation vectors 
to accelerate the convergence of the pressure solution in reservoir simulation.  \\

 \newpage

 \bibliographystyle{unsrt}
 \newpage
 \bibliography{research}
% 
\newpage
\newpage
% \appendix
% \section{List of notation}\label{a1}
% 
% \begin{table}[!h]
% \centering
% \begin{tabular}{c l l }
% \hline
% Symbol & Quantity & Unit \\[0.5ex]
% \hline
% $\phi$ & Rock porosity&   \\
%  $\mathbf{K}$& Rock permeability&  $Darcy$ $(D)$ \\
%  $c_r$& Rock compressibility&  $Pa^{-1}$ \\
% $\mathbf{v}$ & Darcy's velocity& $ m/d$ \\
%  $\alpha$& Geometric factor&   \\
% $\rho$ &Fluid density &  $kg/m^3$ \\
%  $\mu$&Fluid viscosity & $Pa \cdot s$   \\
% ${p}$  &Pressure &  $Pa$ \\
% $g$  &Gravity &  $m/s^2$ \\
% $d$ & Reservoir depth&  $m$ \\
% $c_l$ &Liquid compressibility &  $Pa^{-1}$ \\
% $q$ &Sources &   \\
%  %$T$&Temperature &   \\
%  & &   \\
%  & &   \\
%  & &   \\
%  & &   \\
%  & &   \\
%  & &   \\
%  & &   \\
%  & &   \\
%  & &   \\
%  & &   \\
%  & &   \\
%  & &   \\
%  & &   \\
% %  & &   \\
% \hline
% \end{tabular}\label{table:symbols}
% \caption{Notation}
% \end{table}
% \newpage
% \section{Stopping criteria}\label{a2}
% When we use an iterative method, we always want that our approximation is close enough 
% to the exact solution. In other words, we want that the error \cite[pag. 42]{Saad03}: 
% $$||\mathbf{e}^k||_2=||\mathbf{x}-\mathbf{x}^k||_2,$$ or the relative error: 
% $$\frac{||\mathbf{x}-\mathbf{x}^k||_2}{||\mathbf{x}||_2},$$is small. \\
% When we want to chose a stopping criteria, we could think that the relative error is a
% good candidate, but is has the disadvantage that we need to know the exact solution to compute it.
% What we have instead is the residual $$\mathbf{r}^k=\mathbf{b}-\mathbf{A}\mathbf{x}^k,$$ 
% that is actually computed in each iteration of the CG method. There is a relationship between the 
% error and the residual that can help us with the choice of the stopping criteria.
% $$\frac{||\mathbf{x}-\mathbf{x}^k||_2}{||\mathbf{x}||_2}\leq \kappa_2(A)\frac{||\mathbf{r}^k||_2}{||\mathbf{b}||_2}.$$
% With this relationship in mind, we can choose the stopping criteria as an $\epsilon$ for which
% $$ \frac{||\mathbf{r}^k||_2}{||\mathbf{b}||_2}\leq \epsilon.$$
% But we should keep to have in mind the condition number of the matrix $\mathbf{A}$, because the relative error will be bounded by:
% $$\frac{||\mathbf{x}-\mathbf{x}^k||_2}{||\mathbf{x}||_2}\leq \kappa_2(A) \epsilon.$$
% \section{Singular Value Decomposition for POD}\label{a3}
% If we perform SVD in $\mathbf{X}$, we have \\
% $\mathbf{X}=\mathbf{U}\Sigma \mathbf{V}^T, \qquad \mathbf{U}\in\mathbb{R}^{n \times n},\qquad \mathbf{\Sigma}\in\mathbb{R}^{n \times m}, \qquad \mathbf{V}\in\mathbb{R}^{m \times m}.$\\
% Then we have 
% \begin{itemize}
% \begin{minipage}{.5\textwidth}
% \item[]  $\mathbf{R}= \mathbf{X}\mathbf{X}^T$
%  \item[] $\quad=\mathbf{U}\Sigma \mathbf{V}^T(\mathbf{U}\Sigma \mathbf{V}^T)^T$
%  \item[] $\quad=\mathbf{U}\Sigma \mathbf{V}^T\mathbf{V}\Sigma ^T\mathbf{U}^T,$ $\mathbf{V}^T\mathbf{V}=\mathbf{I}$ 
%   \item[] $\quad=\mathbf{U}\Lambda \mathbf{U}^T$, 
%   $\Lambda=\Sigma\Sigma ^T \in\mathbb{R}^{n \times n}$ 
% \end{minipage}%
% \begin{minipage}{.4\textwidth}
% \item[] $\mathbf{R}^T= \mathbf{X}^T\mathbf{X}$
%  \item[] $\quad=(\mathbf{U}\Sigma \mathbf{V}^T)^T\mathbf{U}\Sigma \mathbf{V}^T$
%  \item[] $\quad=\mathbf{V}\Sigma ^T\mathbf{U}^T\mathbf{U}\Sigma \mathbf{V}^T,$ $\mathbf{U}^T\mathbf{U}=\mathbf{I}$ 
%   \item[] $\quad=\mathbf{V}\Lambda^T \mathbf{V}^T$, 
%   $\Lambda^T=\Sigma ^T\Sigma \in\mathbb{R}^{m \times m}.$
% \end{minipage}
% \end{itemize}
% $$\mathbf{X}=\mathbf{U}\Sigma \mathbf{V}^T$$
% $$\mathbf{U}=\mathbf{X}\mathbf{V}\Sigma^{-1}$$
% $$\mathbf{U}=\mathbf{X}\mathbf{V}\Lambda^{-\frac{1}{2}}$$
% If we compute $\Lambda^T$, we can compute U as follows:
% $$\mathbf{U}=\mathbf{X}\mathbf{V}(\Lambda^T)^{-\frac{T}{2}}=\mathbf{X}\mathbf{V}(\Lambda^T)^{\frac{1}{2}}$$
% % \newpage
% 



\end{document}